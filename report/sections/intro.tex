The Fast Fourier Transform (FFT) is a fundamental algorithm in signal processing, widely utilized for efficiently computing the discrete Fourier transform (DFT) of a sequence. After being popularized by Cooley and Tukey in the 1960s, the FFT has become a cornerstone in various scientific and engineering applications, including image processing, audio analysis, telecommunications, and many others. 

This report provides a comprehensive overview and analysis of our work regarding FFTs, discussing key concepts, algorithms, possible applications, and considerations in the context of some specific implementations. In this project both FFT and DFT algorithms were implemented using different approaches and numerous solutions were used to improve their performance, including parallelization with both OpenMP and CUDA. Other than parallelization, we explored possible applications of the FFT, implementing our own custom and simplified version of the JPEG algorithm. Finally, we implemented two versions of the 9/7 discrete wavelet transform and applied it to image denoising. Our code is available at the following repository: \url{https://github.com/AMSC22-23/FFT-Pesce-Miotti-Allahakbari}.