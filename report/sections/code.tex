\subsection{Dependencies}
The project uses the C++20 standard and has the following dependencies:
\begin{itemize}
    \item CMake version 3.0.0 or higher.
    \item CUDA Toolkit version 12.3 or higher. 
    \item OpenMP version 4.5 or higher.
    \item OpenCV version 4.5.4 or higher. In Ubuntu, the library is available via \texttt{sudo apt install libopencv-dev}.
\end{itemize}
Note that older versions of the tools might not be supported. In particular, the code does not compile when using CUDA Toolkit version 11.5.

\subsection{Compiling}
To compile, run:
\begin{lstlisting}[language=Bash]
    mkdir build
    cd build
    cmake .. [Flags]
    make
\end{lstlisting}
Where [Flags] are:
\begin{itemize}
    \item \texttt{-DCMAKE\_CUDA\_ARCHITECTURES=XX}. For Turing architectures \texttt{XX}=75, for Ampere \texttt{XX}=80,86,87, for Lovelace \texttt{XX}=89 and for Hopper \texttt{XX}=90. 
    \item \texttt{-DCMAKE\_BUILD\_TYPE=YY}, where \texttt{YY} is either \texttt{Debug} or \texttt{Release}, compiling the program in debug and release modes respectively. Both build modes use the flags \texttt{-Wall -Wextra}, but the former includes debug symbols and uses default optimizations, while the latter uses multiple optimization flags, including \texttt{-Ofast}. 
\end{itemize}

\subsection{Running}
To run the program, run \texttt{./fft [args]} while in the \texttt{build} directory. The first argument in [args] is mandatory and it should be \texttt{fft}, \texttt{compression}, \texttt{cuda} or \texttt{wavelet}. Based on the first argument, a different function is called and the program behaves differently.

\subsubsection{FFT}

\subsubsection{Compression}

\subsubsection{CUDA}

\subsubsection{Wavelet}